% Introdução
%--------------------------------------
\chapter{Introdução}


O presente documento tem como objetivo relatar a minha experiência no estágio curri-cular feito na empresa The Loop Co. atraves da unidade curricular "Projeto ou Estágio".
O estágio decorreu durante o segundo semestre do ano letivo 2023/24 e iniciou-se a 15 de fevereiro de 2024 e terminou a *PREENCHER*.

Através deste capítulo irá ser feita uma introdução da entidade acolhedora, The Loop Co., e dos objetivos que se pretendiam atingir durante o estágio, tal como da entidade que me permitiu obter conhecimentos na área de Engenharia Informática, o Instituto Superior de Engenharia de Coimbra (ISEC).

\section{The Loop Co.}
\begin{figure}[h]
    \centering
    \includegraphics[width=0.5\textwidth]{Imagens/logotipo.png} % Use your image file name and path
    \caption{Logotipo The Loop Co.}
    \label{fig:logotipo_the_loop_co}
\end{figure}

A The Loop Co. é uma empresa tecnológica fundada em 2016 com o projeto de eco-nomia circular BookinLoop, sendo uma empresa pioneira em Portugal na reutilização de manuais escolares. Desde então, a empresa expandiu suas atividades para duas áreas principais: economia circular e tecnologia.

A empresa colabora com grandes parceiros nacionais e internacionais, incluindo Sonae, a Cruz Vermelha Portuguesa, o CERN e a Agência Espacial Europeia, em projetos inovadores de desenvolvimento de negócios com foco em tecnologia verde.

A tecnologia verde, ou "green tech", refere-se a tecnologias voltadas para a preservação ambiental e redução do impacto humano na Terra.

Depois de lançar o Book in Loop e BabyLoop, a The Loop Co. percebeu a oportunidade de criar um sistema modular de comércio eletrónico, o LoopOS, que permite setores diversos ingressarem no universo da economia circular rapidamente e de uma forma muito fácil.

*FALAR SOBRE A THE LOOP E A SUA ASSOCIACAO COM O ZEROO*

\section{Do Zeroo}

Com a criação da Zeroo Smart Packaging, dedicada ao futuro do granel e das embala-gens inteligentes, foi lançado o projeto Zero Cups, um conceito de embalagens inteli-gentes e circuláveis para revolucionar as compras a granel, o qual irá ser integrado num dos maiores retalhistas nacionais.

*DESENVOLVER MAIS*

\section{Instituto Superior de Engenharia de Coimbra (ISEC)}
\begin{figure}[h]
    \centering
    \includegraphics[width=0.5\textwidth]{Imagens/isec_logotipo.png} % Use your image file name and path
    \caption{Logotipo Instituto Superior de Engenharia de Coimbra}
    \label{fig:logotipo_isec}
\end{figure}

O Instituto Superior de Engenharia de Coimbra (ISEC), fundado em 1974, é uma das instituições do Instituto Politécnico de Coimbra (IPC). Geograficamente localizado na cidade de Coimbra, na região central de Portugal, o ISEC é especializado em tecnologia e engenharia, proporcionando uma formação de qualidade superior para o exercício das profissões de engenharia.

A oferta educacional do ISEC inclui cursos técnicos superiores profissionais (CTeSP), licenciaturas, pós-graduações e mestrados.

Uma das características mais marcantes do ISEC é o enfoque prático na abordagem educacional, que diferencia a instituição de outras escolas. Esta abordagem prática oferece aos alunos um conhecimento mais direto dos temas abordados, preparando-os de forma mais eficaz para o mercado de trabalho.

\newpage
\section{Zero Cups: Mobile development}

A proposta de estágio apresentada (Anexo A: Proposta de Estágio) referia como princi-pal objetivo desenvolvimento de um conjunto de aplicações móveis (nativas para An-droid ou iOS) e/ou híbridas (React-Native ou outra alternativa proposta), pelo sendo o propó-sito deste estágio trabalhar nos desenvolvimentos inerentes a estas aplicações. 

O presente projeto/estágio pretende atingir os seguintes objetivos genéricos:

\begin{itemize}
    \item Integração do estagiário em ambiente profissional, dando a este a oportunidade de participar e ajudar
    ativamente projetos reais e com impacto;
    \item Oferecer ao estagiário a oportunidade de sentir que o seu contributo estaá a fazer a diferença no desenvolvimento do projeto e no crescimento de uma equipa;
    \item Fomentar o espírito crítico através da participação ativa nas reuniões de equipa, onde aspetos como o rumo, tecnologias e arquitetura da infraestrutura são discu-tidos e delineados;
    \item Dar a perceber a realidade e dinâmica do dia-a-dia de uma empresa em rápido crescimento, de modo a
    ajudar o estagiário a adaptar-se a uma mudança rápida de cenário e objetivos;
    \item Conhecer e interagir com as tecnologias e ferramentas utilizadas internamente e quais as dificuldades e
    vantagens que estas podem apresentar;
    \item Ajudar na procura e implementação de novas tecnologias que possam vir a ser benéficas para a empresa.
\end{itemize}

O projeto/estágio consistirá nas seguintes atividades e respetivas tarefas:

\begin{itemize}
    \item T1 Setup - Ambientação do aluno à empresa e aos métodos de trabalho da mesma.
    \item T2 Estudo Inicial - Realização de alguma investigação acerca das ferramentas e tecnologias a utilizar, bem como algumas formações relevantes. Adicionalmente serão implementados pequenos desafios que ajudem a desenvolver as competên-cias iniciais para começar o projeto do estágio.
    \item T3 Análise de requisitos- Integração no projeto e análise dos requisitos a imple-mentar.
    \item T4 Implementação - Planeamento e implementação das funcionalidades requisi-tadas.
    \item T5 Elaboração do relatório final e preparação para a defesa - Escrita do relatório final de estágio e documentação associada.
\end{itemize}

\newpage
O plano de escalonamento das tarefas é apresentado em seguida:

\begin{table}[H]
    \centering
    \begin{tabular}{|*{1}{p{5cm}}|*{12}{p{0.4cm}|}}
        \hline
        \textbf{Tarefas} & \multicolumn{2}{c|}{\textbf{N}} & \multicolumn{2}{c|}{\textbf{N+1}} & \multicolumn{2}{c|}{\textbf{N+2}} & \multicolumn{2}{c|}{\textbf{N+3}} & \multicolumn{2}{c|}{\textbf{N+4}} & \multicolumn{2}{c|}{\textbf{N+5}} \\
        \hline
        T1 Setup & \cellcolor[HTML]{C0C0C0} & & & & & & & & & & & \\
        \hline
        T2 Estudo inicial & & \cellcolor[HTML]{C0C0C0} & \cellcolor[HTML]{C0C0C0} & & & & & & & & & \\
        \hline
        T3 Análise de requisitos & & & & \cellcolor[HTML]{C0C0C0} & \cellcolor[HTML]{C0C0C0} & & & & & & & \\
        \hline
        T4 Implementação &  &  &  &  &  & \cellcolor[HTML]{C0C0C0} & \cellcolor[HTML]{C0C0C0} & \cellcolor[HTML]{C0C0C0} & \cellcolor[HTML]{C0C0C0} & \cellcolor[HTML]{C0C0C0} & & \\
        \hline
        T5 Elaboração do relatório final e preparação para o defesa &  &  &  &  &  &  &  &  & \cellcolor[HTML]{C0C0C0} & \cellcolor[HTML]{C0C0C0} & \cellcolor[HTML]{C0C0C0} & \cellcolor[HTML]{C0C0C0} \\
        \hline
    \end{tabular}
    \caption{Calendarização original das tarefas}
\end{table}


Apesar deste ser o planeamento apresentado inicialmente, o mesmo sofreu algumas alterações tendo sido gasto menos tempo nas fases iniciais de Setup, Estudo inicial e Análise de requisitos, sendo a maior parte do tempo focado na parte da implementação, seguindo o seguinte escalonamento: 
*PREENCHER CORRETAMENTE*
\begin{table}[H]
    \centering
    \begin{tabular}{|*{1}{p{5cm}}|*{12}{p{0.4cm}|}}
        \hline
        \textbf{Tarefas} & \multicolumn{2}{c|}{\textbf{N}} & \multicolumn{2}{c|}{\textbf{N+1}} & \multicolumn{2}{c|}{\textbf{N+2}} & \multicolumn{2}{c|}{\textbf{N+3}} & \multicolumn{2}{c|}{\textbf{N+4}} & \multicolumn{2}{c|}{\textbf{N+5}} \\
        \hline
        T1 Setup & \cellcolor[HTML]{C0C0C0} & & & & & & & & & & & \\
        \hline
        T2 Estudo inicial & & \cellcolor[HTML]{C0C0C0} & \cellcolor[HTML]{C0C0C0} & & & & & & & & & \\
        \hline
        T3 Análise de requisitos & & & & \cellcolor[HTML]{C0C0C0} & \cellcolor[HTML]{C0C0C0} & & & & & & & \\
        \hline
        T4 Implementação &  &  &  &  &  & \cellcolor[HTML]{C0C0C0} & \cellcolor[HTML]{C0C0C0} & \cellcolor[HTML]{C0C0C0} & \cellcolor[HTML]{C0C0C0} & \cellcolor[HTML]{C0C0C0} & & \\
        \hline
        T5 Elaboração do relatório final e preparação para o defesa &  &  &  &  &  &  &  &  & \cellcolor[HTML]{C0C0C0} & \cellcolor[HTML]{C0C0C0} & \cellcolor[HTML]{C0C0C0} & \cellcolor[HTML]{C0C0C0} \\
        \hline
    \end{tabular}
    \caption{Calendarização real das tarefas}
\end{table}

\newpage
\section{Metodologias de trabalho}

A metodologia de trabalho adotada pela The Loop Co. é baseada em metodologias ágeis, nomeadamente Scrum. Estas metodologias são utilizadas para a gestão de proje-tos de desenvolvimento de software, permitindo uma maior flexibilidade e adaptação a mudanças.
Como tal, cada semana, representa uma sprint de desenvolvimento. Na segunda-feira existe uma Weekly Meeting, onde são revistos os objetivos da semana anterior e se estes foram atingidos ou não, e são definidos objetivos para a presente semana. 
Depois, existe um momento de planning, onde são definidas e planeadas as tarefas da semana. 

Inicialmente, nos restantes dias da semana existiam Daily Meetings, utilizadas para informar a equipa do progresso nas tarefas de cada elemento, e também para identificar possíveis problemas que poderiamos estar a encontrar no desenvolvimento. Posterior-mente, estas reuniões passaram a ser realizadas apenas as quartas e sextas-feiras, visto que a maioria dos elementos da equipa se encontrava no escritório e iam partilhando informações acerca das suas tarefas.

No final da semana, era realizada uma reunião de retrospectiva, onde eram analisados os pontos positivos e negativos da semana, e eram discutidas possíveis melhorias para a semana seguinte.

\section{Estrutura do relatório}
*PREENCHER MAIS TARDE*