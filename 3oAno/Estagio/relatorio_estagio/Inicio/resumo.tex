
% Resumo
%--------------------------------------
\vspace*{45pt}
\begin{flushleft}
	{\Large \textbf{\scshape{Resumo}}}
\end{flushleft}
\vspace*{10pt}


Este documento tem como objetivo descrever e relatar a minha experiência no estágio curricular realizado no âmbito da unidade curricular
"Projeto ou Estágio" da Licencia-tura em Engenharia Informática, no ramo de Desenvolvimento de Aplicações, que decorreu na empresa
The Loop Co. no segundo semestre do ano letivo de 2023/2024.

O estágio teve como principal objetivo o desenvolvimento de uma aplicação para a empresa * Zeroo Cups *. Esta aplicação consiste numa
despensa digital, que inicialmen-te teria sido idealizada para ser desenvolvida em React-Native, mas que mais tarde foi decidida a ser 
implementada em Ruby on Rails, com foco especial na experiência do utilizador em mobile.

Esta aplicação denominada por Zeroo Pantry, tem como foco a gestão dos Zeroo Cups que o utilizador possui, permitindo a este adicionar
novos cups através do scan de um QR Code, ou através da inserção manual do mesmo, e a partir daí, o utilizador poderá gerir o cup recém adicionado,
tão bem como outros cups que já possua. A aplicação também oferece a possibilidade de libertar o cup, querendo isto dizer que a partir do momento em que o cup foi libertado
não estará mais associado ao utilizador e o seu último evento irá para o histórico,
e de "Esvaziar e guardar", querendo isto dizer que o cup continuará na despensa mas vazio e o evento que este teria será passado para o histórico.
Também conta com as funcionalidades de consultar o histórico de eventos do utilizador e dos seus cups e de consultar informações sobre os produtos associados a cada cup.

A Pantry, como é comummente chamada, foi desenvolvida toda ela na linguagem Ruby on Rails, com recurso a uma base de dados PostgreSQL, e com a utilização de
Hotwire, mais concretamente as ferramentas Turbo e Stimulus, para a criação de uma experiência de utilizador mais fluída e rápida, sem a necessidade de recarregar a página.
Relativa-mente aos componentes visuais da aplicação foram utilizados ViewComponents, que consiste num framework para criar componentes reutilizáveis e modulares inspirado no framework ReactJS.


\vspace*{20pt}

\noindent \textbf{Palavras-chave}:
Ruby on Rails, PostgreSQL, Hotwire, Turbo, Stimulus, ReactJS, View-Components.

